This chapter documents the reference implementation of \tbas.

\section{Resolving Variables}

When a variable is \code{resolve}d, an object with instance of \code{BasicVar} is returned. A \code{bvType} of Javascript value is determined using \code{JStoBASICtype}.

\begin{tabulary}{\textwidth}{R|LL}
Typical User Input & TYPEOF(\textbf{Q}) & Instanceof \\
\hline
\code{\textbf{Q}=42.195} & {\ttfamily num} & \emph{primitive} \\
\code{\textbf{Q}=42>21} & {\ttfamily boolean} & \emph{primitive} \\
\code{\textbf{Q}="BASIC!"} & {\ttfamily string} & \emph{primitive} \\
\code{\textbf{Q}=DIM(12)} & {\ttfamily array} & Array (JS) \\
\code{\textbf{Q}=1 TO 9 STEP 2} & {\ttfamily generator} & ForGen \\
\code{DEFUN \textbf{Q}(X)=X+3} & {\ttfamily usrdefun} & BasicAST \\
\end{tabulary}

\subsection*{Notes}
\begin{itemlist}
\item \code{TYPEOF(\textbf{Q})} is identical to the variable's bvType; the function simply returns \code{BasicVar.bvType}.
\item Do note that all resolved variables have \code{troType} of \code{Lit}, see next section for more information.
\end{itemlist}

\section{Unresolved Values}

Unresolved variables has JS-object of \code{troType}, with \emph{instanceof} \code{SyntaxTreeReturnObj}. Its properties are defined as follows:

\begin{tabulary}{\textwidth}{RL}
Properties & Description \\
\hline
{\ttfamily troType} & Type of the TRO (Tree Return Object) \\
{\ttfamily troValue} & Value of the TRO \\
{\ttfamily troNextLine} & Pointer to next instruction, array of: [\#line, \#statement] \\
\end{tabulary}

Following table shows which BASIC object can have which \code{troType}:

\begin{tabulary}{\textwidth}{RLL}
BASIC Type & troType \\
\hline
Any Variable & {\ttfamily lit} \\
Boolean & {\ttfamily bool} \\
Generator & {\ttfamily generator} \\
Array & {\ttfamily array} \\
Number & {\ttfamily num} \\
String & {\ttfamily string} \\
DEFUN'd Function & {\ttfamily internal\_lambda} \\
Array Indexing & {\ttfamily internal\_arrindexing\_lazy} \\
Assignment & {\ttfamily internal\_assignment\_object} \\
\end{tabulary}

\subsection*{Notes}
\begin{itemlist}
\item All type that is not \code{lit} only appear when the statement returns such values, e.g. \code{internal\_lambda} only get returned by DEFUN statements as the statement itself returns defined function as well as assign them to given BASIC variable.
\item As all variables will have \code{troType} of \code{lit} when they are not resolved, the property must not be used to determine the type of the variable; you must \code{resolve} it first.
\item The type string \code{function} should not appear outside of TRO and \code{astType}; if you do see them in the wild, please check your JS code because you probably meant \code{usrdefun}.
\end{itemlist}

\section{Lambda Variables}

Lambda expressions have bound variables: $\lambda x . E$ has bound variable of $x$ and this expression can be written in \tbas\ as \code{[X]$\sim\!>$E}.

However, in the creation and execution of a syntax tree, the bound variables must be able to be properly substituted, but at the same time Closure on \tbas\ can have multiple parameters. How do we solve the substitution and name collision problem?

In 1972, dutch mathematician Nicolaas de Bruijn invented \emph{de Bruijn Indexing}, a system we use to solve the forementioned problems. In de Bruijn Indexing, the innermost bound variable has index of zero\footnote{Smallest number in the set of natural numbers, which can be zero or one depending on your definition of natural numbers; we've obviously chose zero.} and outer variables have greater indices. For example, $\lambda x . \lambda y . \lambda z . x\ z (y\ z)$ is represented as $\lambda\ \lambda\ \lambda\ 2\ 0\ (1\ 0)$

Since closures in \tbas\ can have multiple bound variables instead of just one as in lambda calculus, we deploy a modified version of the indexing:

\begin{align*}
recIndex &:= \text{index of recursion depth}\\
ordIndex &:= \text{reverse order of a bound variable within a level}\\
index &:= \text{array of}\ [recIndex, ordIndex]
\end{align*}

And consider following example code:

\begin{lstlisting}
[X,Y]~>[C]~>ZIP(C,FILTER([M]~>C,ZIP(X,Y)))
\end{lstlisting}

In this code, variables will have following indices:

\begin{tabulary}{\textwidth}{cC|cC|cC}
Variable & Index & Variable & Index & Variable & Index \\
\hline
C {\condensedfont (in filter of M)} & $[1,0]$ & X & $[1,1]$ & M & $[0,0]$ \\
C {\condensedfont (in outer ZIP)} & $[0,0]$ & Y & $[1,0]$ & \ & \ \\
\end{tabulary}

and the program tree would look like this:

{\centering
\begin{minipage}{.5\textwidth}
  \centering
  \includegraphics[width=\linewidth]{lambdaindexing}
  \captionof{figure}{Variable Names Tree}
  \label{fig:lambdaraw}
\end{minipage}%
\begin{minipage}{.5\textwidth}
  \centering
  \includegraphics[width=1.108\linewidth]{lambdaindexing2}
  \captionof{figure}{Variable Indices Tree}
  \label{fig:lambdaindexed}
\end{minipage}
}

As you can clearly observe, there are two \code{C}s, both must refer to the same bound variable but are in different depth. In order to satisfy the constraint, two \code{C}s get different indices. If you were to change \code{[M]} into \code{[C]}, however, the inner \code{C} will get index of $[0,0]$, which points to its immediate \code{C} and it's different \code{C} than one referred in the outer \code{ZIP}, even if the ZIP's \code{C} has identical index of $[0,0]$.

In \code{basic.js}, a variable \code{lambdaBoundVars} will contain such variables, and gets reused in different contexts.
