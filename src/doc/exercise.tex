This chapter contains little more advanced but still short example programs. Wonder how these programs work? Go figure!

\section{Quick Sort}

Prerequisites: Array, Function, Recursion, Closure

\begin{lstlisting}
10 QSORT = [XS] ~> IF LEN(XS) < 1 THEN NIL ELSE 
    QSORT(FILTER([X] ~> X <  HEAD XS, TAIL XS)) # HEAD(XS)!NIL # 
    QSORT(FILTER([X] ~> X >=HEAD XS, TAIL XS))
100 L={7,9,4,5,2,3,1,8,6}
110 PRINT L
120 PRINT QSORT(L)
\end{lstlisting}

\section{Fast Fibonacci Sequence}

Prerequisites: Array, Function, Recursion, Closure, Monad

\begin{lstlisting}
10 FIB_=[N,M] ~> IF LEN(MJOIN(M)) >= N THEN HEAD(MJOIN(M)) ELSE
    FIB_(N,M >>= ([XS] ~> MRET((XS(0) + XS(1)) ! XS)))
11 FIB = [N] ~> FIB_(N, MRET({1,1}))
100 FOR K = 1 TO 10
110 PRINT FIB(K);" ";
120 NEXT
\end{lstlisting}

\section{Count The Length of Chunks}

Prerequisites: Array, Function, Recursion

\begin{lstlisting}
10 DEFUN F(STR,XS) = IF (XS == UNDEFINED) THEN F(STR, {})
   ELSE IF (LEN(STR) == 0) THEN XS
   ELSE F(RIGHT(STR, LEN(STR) - 1),
      IF (LEN(XS) == 0) THEN XS ~ {LEFT(STR, 1), 1}
      ELSE IF (XS(LEN(XS) - 1, 0) >< LEFT(STR, 1)) THEN XS ~ {LEFT(STR, 1), 1}
      ELSE INIT(XS) ~ {LEFT(STR, 1), 1 + XS(LEN(XS) - 1, 1)}
   )
100 PAIRS = F("aaaabbbccaxyzzy")
110 PRINT PAIRS
\end{lstlisting}
