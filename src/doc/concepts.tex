\quad
\chapterprecishere{``Caution! Under no circumstances confuse the adjective \emph{basic} with the noun \emph{BASIC}, except under confusing circumstances!''\par\raggedleft --- \textup{\tbas\ Reference Manual, \theedition}\footnote{Original quotation: Donald R. Woods, James M. Lyon, \emph{The INTERCAL Programming Language Reference Manual}}}

This chapter describes the basic concepts of the \tbas\ language.


\section{Values and Types}
\label{valuesandtypes}

BASIC is a \emph{Dynamically Typed Language}, which means variables do not know which group they should barge in; only values of the variable do. In fact, there is no type definition in the language: \emph{we do want our variables to feel awkward.}

There are eight basic types: \emph{number}, \emph{boolean}, \emph{string}, \emph{array}, \emph{generator}, \emph{function}, \emph{monad} and \emph{undefined}.

\emph{Number}\index{number (type)} represents real (double-precision floating-point or actually, \emph{rational}) numbers. Operations on numbers follow the same rules of the underlying virtual machine\footnote{If you are not a computer person, disregard.}, and such machines must follow the IEEE 754 standard\footnote{ditto.}. 

\emph{Boolean}\index{boolean (type)} is type of the value that is either \codebf{TRUE} or \codebf{FALSE}. Number \codebf{0} and \codebf{FALSE} make the condition \emph{false}. When used in numeric context, \codebf{FALSE} will be interpreted as 0 and \codebf{TRUE} as 1.

\emph{String}\index{string (type)} represents immutable\footnote{Cannot be altered directly.} sequences of bytes.

\emph{Array}\index{array (type)} represents a collection of numbers in one or more dimensions.

\emph{Generator}\index{generator (type)} represents a value that automatically counts up/down whenever they have been called in FOR--NEXT loop.

\emph{Functions}\index{function (type)} are, well\ldots\ functions, especially user-defined ones. Functions are \emph{type} because some built-in functions will actually take \emph{functions} as arguments.

\emph{Monads}\index{monad (type)} are a type that contains some value and follows monad laws. The term Monad refers to any object that satisfies these requirements, however, in \tbas, only one monad type does that (and is useful to you): value-monad.

\emph{Undefined}\index{undefined (type)} represents a value that holds nothing. A function's unspecified arguments, when examined, have this type and are equal to \code{UNDEFINED}. User-defined functions are free to have \code{UNDEFINED} as its arguments, but only the highly limited set of built-in functions will accept one.

\section{Control Flow}

A program is executed starting with its lowest line number. Statements on a line are executed from left to right. When all Statements are finished execution, next lowest line number will be executed. Control flow functions can modify this normal flow.

You can dive into other lines in the middle of the program with \code{GOTO}. The program flow will continue normally at the new line \emph{and it will never know ya just did that}.

If you want less insane jumping, \code{GOSUB} is used to jump to a subroutine. Subroutine is a little section of a code that serves as a tiny program inside of a program. \code{GOSUB} will remember from which statement in the line you have came from, and will return your program flow to that line when \code{RETURN} statement is encountered. (of course, if \code{RETURN} is used without \code{GOSUB}, program will raise some error) Do note that while you can reserve some portion of a program line as a \code{subroutine}, \tbas\ does not provide local variables and whatnot\footnote{If you really need some local variables, use the black magic of Monad.} as all variables in BASIC are global, and you can just \code{GOTO} out of a subroutine to anywhere you desire and wreak havoc \emph{if you really wanted to}.

The \code{ON} statement provides alternative branching construct. You can enter multiple line numbers and let your variable (or mathematical expression) to choose which index of line numbers to \code{GOTO}- or \code{GOSUB} into.

The \code{IF-THEN-ELSE} lets you to conditionally select which of the two branches to execute.

The \code{FOR-NEXT} lets you to loop a portion of a program while automatically counting your chosen variable up or down.
