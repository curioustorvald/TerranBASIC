\label{functions}

Functions are a form of expression that may taks input arguments surrounded by parentheses. Most of the traditional BASIC \emph{statements} that does not return a value are \emph{functions} in \tbas , and like those, while \tbas\ functions can be called without parentheses, it is highly \emph{discouraged} because of the ambiguities in syntax. \textbf{Always use parentheses on function call!}

\section{Mathematical}

    \subsection{ABS}
        \index{ABS (function)}\codeline{Y \textbf{= ABS(}X\textbf{)}}\par
        Returns absolute value of \code{X}.
    \subsection{ACO}
        \index{ACO (function)}\codeline{Y \textbf{= ACO(}X\textbf{)}}\par
        Returns inverse cosine of \code{X}.
    \subsection{ASN}
        \index{ASN (function)}\codeline{Y \textbf{= ASN(}X\textbf{)}}\par
        Returns inverse sine of \code{X}.
    \subsection{ATN}
        \index{ATN (function)}\codeline{Y \textbf{= ATN(}X\textbf{)}}\par
        Returns inverse tangent of \code{X}.
    \subsection{CBR}
        \index{CBR (function)}\codeline{Y \textbf{= CBR(}X\textbf{)}}\par
        Returns cubic root of \code{X}.
    \subsection{CEIL}
        \index{CEIL (function)}\codeline{Y \textbf{= CEIL(}X\textbf{)}}\par
        Returns integer value of \code{X}, truncated towards positive infinity.
    \subsection{COS}
        \index{COS (function)}\codeline{Y \textbf{= COS(}X\textbf{)}}\par
        Returns cosine of \code{X}.
    \subsection{COSH}
        \index{COSH (function)}\codeline{Y \textbf{= COSH(}X\textbf{)}}\par
        Returns hyperbolic cosine of \code{X}.
    \subsection{EXP}
        \index{EXP (function)}\codeline{Y \textbf{= EXP(}X\textbf{)}}\par
        Returns exponential of \code{X}, i.e. $e^X$.
    \subsection{FIX}
        \index{FIX (function)}\codeline{Y \textbf{= FIX(}X\textbf{)}}\par
        Returns integer value of \code{X}, truncated towards zero.
    \subsection{FLOOR, INT}
        \index{FLOOR (function)}\codeline{Y \textbf{= FLOOR(}X\textbf{)}}
        \index{INT (function)}\codeline{Y \textbf{= INT(}X\textbf{)}}\par
        Returns integer value of \code{X}, truncated towards negative infinity.
    \subsection{LEN}
        \index{LEN (function)}\codeline{Y \textbf{= LEN(}X\textbf{)}}\par
        Returns length of \code{X}. \code{X} can be either a string or an array.
    \subsection{LOG}
        \index{LOG (function)}\codeline{Y \textbf{= LOG(}X\textbf{)}}\par
        Returns natural logarithm of \code{X}.
    \subsection{ROUND}
        \index{ROUND (function)}\codeline{Y \textbf{= ROUND(}X\textbf{)}}\par
        Returns closest integer value of \code{X}, rounding towards positive infinity.
    \subsection{RND}
        \index{RND (function)}\codeline{Y \textbf{= RND(}X\textbf{)}}\par
        Returns a random number within the range of $[0..1)$. If \code{X} is zero, previous random number will be returned; otherwise new random number will be returned.
    \subsection{SIN}
        \index{SIN (function)}\codeline{Y \textbf{= SIN(}X\textbf{)}}\par
        Returns sine of \code{X}.
    \subsection{SINH}
        \index{SINH (function)}\codeline{Y \textbf{= SINH(}X\textbf{)}}\par
        Returns hyperbolic sine of \code{X}.
    \subsection{SGN}
        \index{SGN (function)}\codeline{Y \textbf{= SGN(}X\textbf{)}}\par
        Returns sign of \code{X}: 1 for positive, -1 for negative, 0 otherwise.
    \subsection{SQR}
        \index{SQR (function)}\codeline{Y \textbf{= SQR(}X\textbf{)}}\par
        Returns square root of \code{X}.
    \subsection{TAN}
        \index{TAN (function)}\codeline{Y \textbf{= TAN(}X\textbf{)}}\par
        Returns tangent of \code{X}.
    \subsection{TANH}
        \index{TANH (function)}\codeline{Y \textbf{= TANH(}X\textbf{)}}\par
        Returns hyperbolic tangent of \code{X}.

\section{Input}

    \subsection{CIN}
        \index{CIN (function)}\codeline{S \textbf{= CIN()}}\par
        Waits for the user input and returns it.
    \subsection{DATA}
        \index{DATA (function)}\codeline{\textbf{DATA} CONST0 [\textbf{,} CONST1]\ldots}\par
        Adds data that can be read by \code{READ} function. \code{DATA} declarations need not be reacheable in the program flow.
    \subsection{DGET}
        \index{DGET (function)}\codeline{S \textbf{= DGET()}}\par
        Fetches a data declared from \code{DATA} statements and returns it, incrementing the \code{DATA} position.
    \subsection{DIM}
        \index{DIM (function)}\codeline{Y \textbf{= DIM(}X\textbf{)}}\par
        Returns array with size of \code{X}, all filled with zero.
    \subsection{GETKEYSDOWN}
        \index{GETKEYSDOWN (function)}\codeline{K \textbf{= GETKEYSDOWN()}}\par
        Stores array that contains keycode of keys held down into the given variable.\par
        Actual keycode and the array length depends on the machine: in \thismachine , array length will be fixed to 8. For the list of available keycodes, see \ref{implementation}.
    \subsection{INPUT}
        \index{INPUT (function)}\codeline{\textbf{INPUT} VARIABLE}\par
        Prints out \code{? } to the console and waits for user input. Input can be any length and terminated with return key. The input will be stored to given variable.\par
        This behaviour is to keep the compatibility with the traditional BASIC. For function-like usage, use \code{CIN} instead.
    \subsection{READ}
        \index{READ (function)}\codeline{\textbf{READ} VARIABLE}\par
        Assigns data declared from \code{DATA} statements to given variable. Reading starts at the current \code{DATA} position, and the data position will be incremented by one. The position is reset to the zero by the \code{RUN} command.\par
        This behaviour is to keep the compatibility with the traditional BASIC. For function-like usage, use \code{DGET} instead.
        
\section{Output}

    \subsection{EMIT}
        \index{EMIT (function)}\codeline{\textbf{EMIT(} EXPR [\{\textbf{,}|\textbf{;}\} EXPR]\ldots\ \textbf{)}}\par
        Prints out characters corresponding to given number on the code page being used.\par
        \code{EXPR} is numeric expression.
    \subsection{PRINT}
        \index{PRINT (function)}\codeline{\textbf{PRINT(} EXPR [\{\textbf{,}|\textbf{;}\} EXPR]\ldots\ \textbf{)}}\par
        Prints out given string expressions.\par
        \code{EXPR} is a string, numeric expression, or array.\par
        \code{PRINT} is one of the few function that differentiates two style of argument separator: \codebf{;} will simply concatenate two expressions (unlike traditional BASIC, numbers will not have surrounding spaces), \codebf{,} tabulates the expressions.

\section{Program Manipulation}

    \subsection{CLEAR}
        \index{CLEAR (function)}\codeline{\textbf{CLEAR}}\par
        Clears all declared variables.
    \subsection{END}
        \index{END (function)}\codeline{\textbf{END}}\par
        Stops program execution and returns control to the user.
    \subsection{FOR}
        \index{FOR (function)}\codeline{\textbf{FOR} LOOPVAR \textbf{=} START \textbf{TO} STOP [\textbf{STEP} STEP]}
        \codeline{\textbf{FOR} LOOPVAR \textbf{=} GENERATOR}\par
        Starts a FOR--NEXT loop.\par
        Initially, \code{LOOPVAR} is set to \code{START} then statements between the \code{FOR} statement and corresponding \code{NEXT} statements are executed and \code{LOOPVAR} is incremented by \code{STEP}, or by 1 if \code{STEP} is not specified. The program flow will continue to loop around until \code{LOOPVAR} is outside the range of \code{START}--\code{STOP}. The value of the \code{LOOPVAR} is equal to \code{STOP}$+$\code{STEP} when the looping finishes.
    \subsection{FOREACH}
        \index{FOREACH (function)}\codeline{\textbf{FOREACH} LOOPVAR \textbf{IN} ARRAY}\par
        Same as \code{FOR} but fetches \code{LOOPVAR} from given \code{ARRAY}.
    \subsection{GOSUB}
        \index{GOSUB (function)}\codeline{\textbf{GOSUB} LINENUM}\par
        Jumps to a subroutine at \code{LINENUM}. The next \code{RETURN} statements makes program flow to jump back to the statement after the \code{GOSUB}.\par
        \code{LINENUM} can be either a numeric expression or a Label.
    \subsection{GOTO}
        \index{GOTO (function)}\codeline{\textbf{GOTO} LINENUM}\par
        Jumps to \code{LINENUM}.\par
        \code{LINENUM} can be either a numeric expression or a Label.
    \subsection{LABEL}
        \index{LABEL (function)}\codeline{\textbf{LABEL} NAME}\par
        Puts a name onto the line numeber the statement is located. Jumping to the \code{NAME}
        \subsubsection*{Notes}
        \begin{itemlist}
        \item \code{NAME} must be a valid variable name.
        \end{itemlist}
    \subsection{NEXT}
        \index{NEXT (function)}\codeline{\textbf{NEXT}}\par
        Iterates FOR--NEXT loop and increments the loop variable from the most recent \code{FOR} statement and jumps to that statement.
    \subsection{RESTORE}
        \index{RESTORE (function)}\codeline{\textbf{RESTORE}}\par
        Resets the \code{DATA} pointer.
    \subsection{RETURN}
        \index{RETURN (function)}\codeline{\textbf{RETURN}}\par
        Returns from the \code{GOSUB} statement.

\section{String Manipulation}

    \subsection{CHR}
        \index{CHR (function)}\codeline{CHAR \textbf{= CHR(}X\textbf{)}}\par
        Returns the character with code point of \code{X}. Code point is a numeric expression in the range of $[0-255]$.
    \subsection{LEFT}
        \index{LEFT (function)}\codeline{SUBSTR \textbf{= LEFT(} STR \textbf{,} NUM\_CHARS \textbf{)}}\par
        Returns the leftmost \code{NUM\_CHARS} characters of \code{STR}.
    \subsection{MID}
        \index{MID (function)}\codeline{SUBSTR \textbf{= MID(} STR \textbf{,} POSITION \textbf{,} LENGTH \textbf{)}}\par
        Returns a substring of \code{STR} starting at \code{POSITION} with specified \code{LENGTH}.\par
        When \code{OPTIONBASE 1} is specified, the position starts from 1; otherwise it will start from 0. 
    \subsection{RIGHT}
        \index{RIGHT (function)}\codeline{SUBSTR \textbf{= RIGHT(} STR \textbf{,} NUM\_CHARS \textbf{)}}\par
        Returns the rightmost \code{NUM\_CHARS} characters of \code{STR}.
    \subsection{SPC}
        \index{SPC (function)}\codeline{STR \textbf{= SPC(} STR \textbf{,} NUM\_CHARS \textbf{)}}\par
        Returns a string of \code{NUM\_CHARS} spaces.

\section{Array Manipulation}
        
    \subsection{HEAD}
        \index{HEAD (function)}\codeline{K \textbf{= HEAD(}X\textbf{)}}\par
        Returns the head element of the array \code{X}.
    \subsection{INIT}
        \index{INIT (function)}\codeline{K \textbf{= INIT(}X\textbf{)}}\par
        Constructs the new array from array \code{X} that has its last element removed.
    \subsection{LAST}
        \index{LAST (function)}\codeline{K \textbf{= LAST(}X\textbf{)}}\par
        Returns the last element of the array \code{X}.
    \subsection{TAIL}
        \index{TAIL (function)}\codeline{K \textbf{= TAIL(}X\textbf{)}}\par
        Constructs the new array from array \code{X} that has its head element removed.
        
\section{Graphics}

    \subsection{CLS}
        \index{CLS (function)}\codeline{\textbf{CLS}}\par
        Clears text view and moves text cursor to top-left.
    \subsection{PLOT}
        \index{PLOT (function)}\codeline{\textbf{PLOT(} X\_POS \textbf{,} Y\_POS \textbf{,} COLOUR \textbf{)}}\par
        Plots a pixel to the framebuffer of the display, at XY-position of \code{X\_POS} and \code{Y\_POS}, with colour of \code{COLOUR}.\par
        Top-left corner of the pixel will be 1 if \code{OPTIONBASE 1} is specified; otherwise it will be 0.

\section{Meta}

    \subsection{OPTIONBASE}
        \index{OPTIONBASE (function)}\codeline{\textbf{OPTIONBASE} \{\textbf{0}|\textbf{1}\}}\par
        Specifies at which number the array/string/pixel indices begin.
    \subsection{OPTIONDEBUG}
        \index{OPTIONDEBUG (function)}\codeline{\textbf{OPTIONDEBUG} \{\textbf{0}|\textbf{1}\}}\par
        Specifies whether or not the debugging messages should be printed out. The messages will be printed out to the \emph{serial debugging console}, or to the stdout.
    \subsection{OPTIONTRACE}
        \index{OPTIONTRACE (function)}\codeline{\textbf{OPTIONTRACE} \{\textbf{0}|\textbf{1}\}}\par
        Specifies whether or not the line numbers should be printed out. The messages will be printed out to the \emph{serial debugging console}, or to the stdout.
    \subsection{TYPEOF}
        \index{TYPEOF (function)}\codeline{X \textbf{= TYPEOF(} VALUE \textbf{)}}\par
        Returns a type of given value.\par
        \begin{longtable}{*{2}{m{\textwidth}}}\hline
        \endfirsthead
        \endhead

        \endfoot
        \hline
        \endlastfoot
        \centering
        \begin{tabulary}{\textwidth}{rl}
        BASIC Type & Returned Value \\
        \hline
        Number & \ttfamily{num} \\
        Boolean & \ttfamily{bool} \\
        String & \ttfamily{str} \\
        \end{tabulary}
        \begin{tabulary}{\textwidth}{rl}
        BASIC Type & Returned Value \\
        \hline
        Array & \ttfamily{array} \\
        Generator & \ttfamily{generator} \\
        User Function & \ttfamily{usrdefun}
        \end{tabulary}
        \end{longtable}
    
\section{System}

    \subsection{PEEK}
        \index{PEEK (function)}\codeline{BYTE \textbf{= PEEK(} MEM\_ADDR \textbf{)}}\par
        Returns whatever the value stored in the \code{MEM\_ADDR} of the Scratchpad Memory.\par
        Address mirroring, illegal access, etc. are entirely up to the virtual machine which the BASIC interpreter is running on.
    \subsection{POKE}
        \index{POKE (function)}\codeline{\textbf{POKE(} MEM\_ADDR \textbf{,} BYTE \textbf{)}}\par
        Puts a \code{BYTE} into the \code{MEM\_ADDR} of the Scratchpad Memory.

\section{Higher-order Function}

    \subsection{DO}
        \index{DO (function)}\codeline{\textbf{DO(} EXPR0 [\textbf{;} EXPR1]\ldots\ \textbf{)}}\par
        Executes \code{EXPRn}s sequentially.
    \subsection{FILTER}
        \index{FILTER (function)}\codeline{NEWLIST \textbf{= FILTER(} FUNCTION \textbf{,} ITERABLE \textbf{)}}\par
        Returns an array of values from the \code{ITERABLE} that passes the given function. i.e. values that makes \code{FUNCTION(VALUE\_FROM\_ITERABLE)} true.
        \subsubsection*{Parameters}
        \begin{itemlist}
        \item \code{FUNCTION} is a user-defined function with single parameter.
        \item \code{ITERABLE} is either an array or a generator.
        \end{itemlist}
    \subsection{FOLD}
        \index{FOLD (function)}\codeline{NEWVALUE \textbf{= FOLD(} FUNCTION \textbf{,} INIT\_VALUE \textbf{,} ITERABLE \textbf{)}}\par
        Iteratively applies given function with accumulator and the value from the \code{ITERABLE}, returning the final accumulator. Accumulator will be set to \code{INIT\_VALUE} before iterating over the iterable. In the first execution, the accumulator will be set to \code{ACC=FUNCTION(ACC,ITERABLE(0))}, and the execution will continue to remaining values within the iterable until all values are consumed. The \code{ITERABLE} will not be modified after the execution.
        \subsubsection*{Parameters}
        \begin{itemlist}
        \item \code{FUNCTION} is a user-defined function with two parameters: first parameter being accumulator and second being a value.
        \end{itemlist}
    \subsection{MAP}
        \index{MAP (function)}\codeline{NEWLIST \textbf{= MAP(} FUNCTION \textbf{,} ITERABLE \textbf{)}}\par
        Applies given function onto the every element in the iterable, and returns an array that contains such items. i.e. returns tranformation of \code{ITERABLE} of which the transformation is \code{FUNCTION}. The \code{ITERABLE} will not be modified after the execution.
        \subsubsection*{Parameters}
        \begin{itemlist}
        \item \code{FUNCTION} is a user-defined function with single parameter.
        \end{itemlist}
