This is proof that value-monad of the \tbas\ obeys monad laws.

Monad laws are three equations that make sure monads to have sensible behaviours:
\begin{align*}
\text{return}\ a \text{\enskip\basicmbind\enskip}& f &\equiv&\enskip f a\\
m \text{\enskip\basicmbind\enskip}& \text{return} &\equiv&\enskip m\\
m \text{\enskip\basicmbind\enskip}& (\lambda x \rightarrow k\ x \text{\enskip\basicmbind\enskip} h) &\equiv&\enskip (m \text{\enskip\basicmbind\enskip} k) \text{\enskip\basicmbind\enskip} h
\end{align*}
These are referred as \emph{Left identity}, \emph{Right identity} and \emph{Associativity} respectively.

\begin{lstlisting}
10 F=[X]~>X*2 : G=[X]~>X^3 : RETN=[X]~>MRET(X)

100 PRINT:PRINT "First law: 'return a >>= k' equals to 'k a'"
110 K=[X]~>RETN(F(X)) : REM K is monad-returning function
120 A=42
130 KM=RETN(A)>>=K
140 KO=K(A)
150 PRINT("KM is ";TYPEOF(KM);", ";EVALMONAD(KM))
160 PRINT("KO is ";TYPEOF(KO);", ";EVALMONAD(KO))

200 PRINT:PRINT "Second law: 'm >>= return' equals to 'm'"
210 M=MRET(G(42))
220 MM=M>>=RETN
230 MO=M
240 PRINT("MM is ";TYPEOF(MM);", ";EVALMONAD(MM))
250 PRINT("MO is ";TYPEOF(MO);", ";EVALMONAD(MO))

300 PRINT:PRINT "Third law: 'm >>= (\x -> k x >>= h)' equals to '(m >>= k) >>= h'"
310 REM see line 110 for the definition of K
320 H=[X]~>RETN(G(X)) : REM H is monad-returning function
330 M=MRET(69)
340 M1=M>>=([X]~>K(X)>>=H)
350 M2=(M>>=K)>>=H
360 PRINT("M1 is ";TYPEOF(M1);", ";EVALMONAD(M1))
370 PRINT("M2 is ";TYPEOF(M2);", ";EVALMONAD(M2))
\end{lstlisting}

Monad laws are also preserved when arrays are used:

\begin{lstlisting}
10 F=[X]~>RETN(X~LAST(X)*2) : G=[X]~>RETN(X~LAST(X)^3) : RETN=[X]~>MRET(X)

100 PRINT:PRINT "First law: 'return a >>= k' equals to 'k a'"
110 K=[X]~>F(X) : REM K is monad-returning function
120 A=42!NIL
130 KM=RETN(A)>>=K
140 KO=K(A)
150 PRINT("KM is ";TYPEOF(KM);", ";MEVAL(KM))
160 PRINT("KO is ";TYPEOF(KO);", ";MEVAL(KO))

200 PRINT:PRINT "Second law: 'm >>= return' equals to 'm'"
210 M=G(42!NIL)
220 MM=M>>=RETN
230 MO=M
240 PRINT("MM is ";TYPEOF(MM);", ";MEVAL(MM))
250 PRINT("MO is ";TYPEOF(MO);", ";MEVAL(MO))

300 PRINT:PRINT "Third law: 'm >>= (\x -> k x >>= h)' equals to '(m >>= k) >>= h'"
310 REM see line 110 for the definition of K
320 H=[X]~>G(X) : REM H is monad-returning function
330 M=RETN(69!NIL)
340 M1=M>>=([X]~>K(X)>>=H)
350 M2=(M>>=K)>>=H
360 PRINT("M1 is ";TYPEOF(M1);", ";MEVAL(M1))
370 PRINT("M2 is ";TYPEOF(M2);", ";MEVAL(M2))
\end{lstlisting}
